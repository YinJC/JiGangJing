%!TEX TS-program = xelatex
%! TEX encoding = UTF-8 Unicode

%========================================全文布局
\documentclass[12pt,twoside,openany]{book}
\usepackage[screen,paperheight=14.4cm,paperwidth=10.8cm,
left=2mm,right=2mm,top=2mm,bottom=5mm]{geometry}

\usepackage[]{microtype}
\usepackage{graphicx}
\usepackage{amssymb,amsmath}
\usepackage{booktabs}
\usepackage{titletoc}
\usepackage{titlesec}
\usepackage{tikz}
\usepackage{enumerate}
\usepackage{wallpaper}
\usepackage{indentfirst}
%========================================设置字体
\usepackage[CJKnumber]{xeCJK}
\setCJKmainfont[BoldFont={Adobe Heiti Std R}]{Hiragino Sans GB W3}
\setCJKfamilyfont{kai}{Adobe Kaiti Std R}
\setCJKfamilyfont{hei}{Adobe Heiti Std R}
\setCJKfamilyfont{fsong}{Adobe Fangsong Std R}

\newcommand{\kai}[1]{{\CJKfamily{kai}#1}}
\newcommand{\hei}[1]{{\CJKfamily{hei}#1}}
\newcommand{\fsong}[1]{{\CJKfamily{fsong}#1}}

\renewcommand\contentsname{目~录~}
\renewcommand\listfigurename{图~列~表~}
\renewcommand\listtablename{表~目~录~}

%========================================章节样式
\titlecontents{chapter}
[0em]
{}
{\large\CJKfamily{hei}{}}
{}{\dotfill\contentspage}%用点填充
%
\titlecontents{section}
[4em]
{}
{\thecontentslabel\quad}
{}{\titlerule*{.}\contentspage}

\titleformat{\chapter}[display]
	{\CJKfamily{fsong}\Large\centering}
	{\titlerule[1pt]%
	 \filleft%
	}
	{-7ex}
	{\Huge
	 \filright}
	[{\titlerule[1pt]}]

%========================================设置目录
\usepackage[setpagesize=false,
            linkcolor=black,
            colorlinks, %注释掉此项则交叉引用为彩色边框(将colorlinks和pdfborder同时注释掉)
            pdfborder=001   %注释掉此项则交叉引用为彩色边框
            ]{hyperref}

\setlength{\parindent}{2em} %首行缩进
\linespread{1.2}              %行距
\setlength{\parskip}{15pt}    %段距

%========================================页眉页脚
\usepackage{fancyhdr}
\pagestyle{fancy}
\fancyhf{}
\fancyfoot{}
\fancyfoot[LE,RO]{\thepage}
\setlength{\footskip}{6pt}
%========================================标题作者
\title{金刚经}
\author{星云大师\,浅译}
\date{ }

%========================================正文
\begin{document}
\TileSquareWallPaper{1}{TGTamber}%背景图片

\maketitle
\tableofcontents
%\newpage

\noindent
%\chapter*{简介}\addcontentsline{toc}{chapter}{\large\CJKfamily{hei}简介}
\chapter{第一品\ 法会因由分}

如是我闻:一时,佛在舍卫国祇树给孤独园,与大比丘众千二百五十人俱。尔时,世尊食时,著衣持钵,入舍卫大城乞食。于其城中次第乞已,还至本处。饭食讫,收衣钵,洗足已,敷座而坐。

\kai{《金刚般若》这部经是我阿难亲自听到佛陀这样说的:那时候,佛陀住在舍卫国的只树给孤独园中,有一千二百五十位大比丘众随侍左右。有一天,已到了吃饭的时候了。佛陀穿上袈裟,拿著饭钵,带领著弟子们走进舍卫城去乞食。不分贫富不分贵贱,挨家挨户地托钵,乞食后,回到给孤独园中。吃过饭后,佛陀将衣、钵收拾好,洗净了双足,舖好座位便盘腿静坐。}

\chapter{第二品\ 善现启请分}

时长老须菩提在大众中,即从座起,偏袒右肩,右膝著地,合掌恭敬,而白佛言:“希有,世尊,如来善护念诸菩萨,善付嘱诸菩萨。世尊,善男子善女人,发阿耨多罗三藐三菩提心,云何应住,云何降伏其心?”佛言:“善哉,善哉,须菩提,如汝所说,如来善护念诸菩萨,善付嘱诸菩萨。汝今谛听,当为汝说,善男子善女人,发阿耨多罗三藐三菩提心,应如是住,如是降伏其心。”“唯然,世尊,愿乐欲闻。”

\kai{这时,长老须菩提,在大众中站起来,偏袒著右肩,以右膝跪地,双手合拿,虔诚恭敬地向佛陀问道:「世间希有的佛陀!佛陀善于爱护顾念诸菩萨,善于教导付嘱诸菩萨。佛陀!如果有善男子、善女人,已发起无上正等正觉的菩提心,如何才能安住?如何才能降伏妄心?」佛陀嘉许说:「很好!很好!须菩提!正如你所说,佛陀善于爱护顾念诸菩萨,善于教导付嘱诸菩萨。你们现在细心静听,我为你们解说,如何安住菩提心,如何降伏妄想心,善男子、善女人,发了无上正等正觉的菩提心,应该如下所说,如此去安住菩提心,不令忘失;如此去降伏妄想心,令它不再生起。」「是的,佛陀!我们大家都乐意听闻。」}

\chapter{第三品\ 大乘正宗分}

佛告须菩提:“诸菩萨摩诃萨,应如是降伏其心,所有一切众生之类,若卵生、若胎生、若湿生、若化生,若有色、若无色,若有想、若无想、若非有想非无想,我皆令入无余涅槃而灭度之。如是灭度无量无数无边众生,实无众生得灭度者。何以故?须菩提,若菩萨有我相、人相、众生相、寿者相,即非菩萨。”

\kai{佛陀告诉须菩提,「诸位菩萨摩诃萨,应当如此降伏妄心;对所有一切众生,不同生命形态的卵生、湿生、化生;有色身、无色身;有心思想念的、无心思想念的、不是有想不是无想的众生等,都要使他们进入无余涅槃的境界,了断一切苦报、烦恼,渡过生死苦海,到达不生不死之地。如此灭度无量无数无边的众生,其实并不见有一个众生为我所度。这是什么缘故?须菩提!若菩萨妄执有我、人、众生、寿者四相对待别分,以为有个我能化度众生,又见有所谓的众生为我所度,这样就不能称为菩萨了。}

\chapter{第四品\ 妙行无住分}

“复次,须菩提,菩萨于法,应无所住,行于布施。所谓不住色布施,不住声、香、味、触、法布施。须菩提,菩萨应如是布施,不住于相。何以故?若菩萨不住相布施,其福德不可思量。须菩提,于意云何,东方虚空可思量不?”“不也,世尊。”“须菩提,南西北方,四维上下虚空,可思量不?”“不也,世尊。”“须菩提,菩萨无住相布施,福德亦复如是,不可思量。须菩提,菩萨但应如所教住。”

\kai{「再者,菩萨了知一切诸法其性本空,为因缘聚灭会合,所以于世间所有的万事万物,都应无所执著,以此无住法中,修行布施,利益众生。也就是六根清净,不住色声香味触法等六尘,而去行布施。这是什么缘故?若菩萨修行无相布施,没有布施的我,受布施的人,所布施的物,当然布施后更不存求报的念头,这种三轮体空,无相而施的福德是不可思量。「须菩提!你认为东方的虚空可以思量得到吗?」「不可思量的,佛陀!」「须菩提!那么南西北方四维上下的虚空,可以思量得到吗?」「不可思量的,佛陀!」「须菩提!菩萨因体悟三轮体空,不执著事相而行布施,其所得的福德,也和十方虚空一样,是不可思量。须菩提!菩萨只要依著我的教法修行,自然能令妄心不起,真正安住于清净的菩提本心。」}

\chapter{第五品\ 如理实见分}

“须菩提,于意云何,可以身相见如来不?”“不也,世尊,不可以身相得见如来。何以故?如来所说身相,即非身相。”佛告须菩提:“{\color{red}{凡所有相,皆是虚妄。若见诸相非相,即见如来。}}”

\kai{「须菩提!你认为可以从身相见到佛陀吗?」「不可以的,佛陀!不可以从身相见到佛陀。为什么?因为佛陀所说的身相,指的是色身。色身是地水火风四大假合,是因缘生灭,虚妄不实的,并非真实永存之身。佛陀的真实法身,等如虚空,无所不在。但是法身无相,凡眼是无法亲见,只有明了五蕴假合的幻相,才能亲到佛陀不生不灭的法身。」佛陀告诉须菩提说:「不仅佛身如此,凡是世间所有诸相,都是生灭迁流的相,虚妄不实的。若能了达世间虚妄的本质,就能见到佛陀的法身了。」’}

\chapter{第六品\ 正信希有分}
须菩提白佛言:“世尊,颇有众生,得闻如是言说章句,生实信不?” 佛告须菩提:“莫作是说。如来灭后,后五百岁,有持戒修福者,于此章句,能生信心,以此为实,当知是人,不于一佛、二佛、三四五佛而种善根,已于无量千万佛所种诸善根。闻是章句,乃至一念生净信者,须菩提,如来悉知悉见。是诸众生,得如是无量福德。何以故?{\color{red}是诸众生,无复我相、人相、众生相、寿者相,无法相,亦无非法相。}何以故?是诸众生,若心取相,即为著我、人、众生、寿者;若取法相,即著我、人、众生、寿者。何以故?若取非法相,即著我、人、众生、寿者。是故不应取法,不应取非法。以是义故,如来常说,汝等比丘,知我说法,如筏喻者,法尚应舍,何况非法。”

\kai{须菩须又问道:「佛陀!后世的许多众生,听闻您今日所说的微妙言说、章句,能不能因此而生实信之心?」 佛陀回答须菩提说:「不要这样怀疑;在我灭度后的第五个五百年,若有持守戒律、广修福德的人,能从这些言说章句,体悟无住的实相般若妙义,而生出难得的真实信心。应当知道这些人,不止曾经于一佛、二佛、三、四、五佛所种植诸善根,其实他们已于多生劫来,奉事诸佛,种诸善根,现世闻说大乘无住的般若真理,乃至只是一念之间生起清净信心的人,须菩提!如来是无所不知无所不见的,这些善根众生,是会得到无限福德的。 「这是什么道理呢?是因为这些善根众生,不再妄执有我、人、众生、寿者四相的对待分别,不会执著有为的生灭法相,也不会执著无为的空寂法相。也没有不是诸法的执相。如此则心无所住,而修无相之行,故获功德广大。 「这是什么缘故呢?如果众生一念心,于相上有所取著,则会落于我、人、众生、寿者四相的对待分别中。同样地,若众生执著种种法相,即于我、人等四相有所取著。若又执著无法相,则同样地也会落于我、人等四相的对待分别中。「因为取法则滞于有,以为有实有的生灭法相可离;取非法则泥于空,以为又有空寂的非法相可证得,不能与空理相契,所以法相与非法相都不该执取。 因此,如来常说:「你们诸位比丘应当知晓,我所说的佛法,就如同那渡人到岸的舟楫,到达彼岸之后,即应弃舟登岸,不可揹负不舍。所以,未悟道时,须依法修持,悟道后就不该执著于法,至于那偏执于非法的妄心,更是应当舍去。」}

\chapter{第七品\ 无得无说分}
“须菩提,于意云何,如来得阿耨多罗三藐三菩提耶?如来有所说法耶?” 须菩提言:“如我解佛所说义,无有定法名阿耨多罗三藐三菩提,亦无有定法如来可说。何以故?如来所说法,皆不可取、不可说,非法、非非法。所以者何?一切贤圣,皆以无为法而有差别。”

\kai{「须菩提!你认为如来已证得了无上正等正觉吗?如来有所说法吗?」须菩提回答说:「就我所了解佛陀说法的义理,是没有一定的法可以叫做无上正等正觉,也没有固定的法,为如来所说。什么缘故呢?因为如来所说的法,都是为了众生修行及开悟众生而假设的方便之法,不可以执取,般若的实相,是无法以语言诠释的,执着实有的菩提可得,也不可执著没有菩提正觉,落于有和空,都是错误的。「这是什么缘故呢?因为没有一定的法名为菩提,一切贤圣,也都是依寂灭的无为法而修,因证悟的深浅不同,才产生有三贤十圣等阶位的差别。}

\chapter{第八品\ 依法出生分}
“须菩提,于意云何?若人满三千大千世界七宝,以用布施,是人所得福德,宁为多不?须菩提言:“甚多,世尊。何以故?是福德,即非福德性,是故如来说福德多。”“若复有人,于此经中,受持乃至四句偈等,为他人说,其福胜彼。何以故?须菩提,一切诸佛,及诸佛阿耨多罗三藐三菩提法,皆从此经出。须菩提,所谓佛法者,即非佛法。”

\kai{「 须菩提!譬若有人用盛满三千大千世界的七宝去布施结缘,你认为这人所获得的福德果报,多不多呢?」 须菩提回答道:「很多,佛陀!为什么?因为七宝布施,所获得的是世间有相的福德,所以佛陀说福德多;如果从性上说,没有所谓福德的名称,哪里有多和少可说呢?佛陀不过是随顺世俗,说七宝的布施,所获的福德是很多。」 「如果又有一人,能够信受奉持此部经,即使短至受持其中四句偈等,又能够为他人解说,那么,他所得的福德果报更要胜过布施七宝的人。什么缘故呢?须菩提!因为十方一切诸佛,都从此经出生,此般若法为诸佛之母;又一切无上正等正觉法,亦从此经出生,此经又为诸法之母。因此,如果没有此经,也就没有十方一切诸佛,以及成佛的无上正等正觉法。 「须菩提!所谓的佛法,不过依俗谛而立的假名,并非就是真实的佛法,因为众生有凡圣迷悟的分别执著,佛陀为了开悟众生,不得不方便言说。若以法性毕竟空而言,求诸佛的名字称尚不可得,还有什么叫做成佛的无上正等正觉之法呢?」}

\chapter{第九品\ 一相无相分}
“须菩提,于意云何,须陀洹能作是念,我得须陀洹果不?” 须菩提言:“不也,世尊。何以故?须陀洹名为入流,而无所入,不入色、声、香、味、触、法,是名须陀洹。”“须菩提,于意云何,斯陀含能作是念,我得斯陀含果不?” 须菩提言:“不也,世尊。何以故?斯陀含名一往来,而实无往来,是名斯陀含。”“须菩提,于意云何?阿那含能作是念,我得阿那含果不?” 须菩提言:“不也,世尊。何以故?阿那含名为不来,而实无不来,是故名阿那含。”“须菩提,于意云何?阿罗汉能作是念,我得阿罗汉道不?” 须菩提言:“不也,世尊。何以故?实无有法名阿罗汉。世尊,若阿罗汉作是念,我得阿罗汉道,即为著我、人、众生、寿者。世尊,佛说我得无诤三昧,人中最为第一,是第一离欲阿罗汉。世尊,我不作是念:‘我是离欲阿罗汉。’世尊,我若作是念:‘我得阿罗汉道’,世尊则不说须菩提是乐阿兰那行者。以须菩提实无所行,而名须菩提是乐阿兰那行。”

\kai{「须菩提!你认为须陀洹会生起这样的心念?『我已证得须陀洹果!』」须菩提回答:「不会的,佛陀!为什么呢?须陀洹的意思是入圣流,而事实上是无所入的,不执着色、声、香、味、触、法等六尘境相,因为心中没有取舍的妄念,不随六尘流转,所以,才叫作须陀洹。」「须菩提!你认为斯陀含会有那样的念头吗?『我已证得斯陀含果!』」「不会的,佛陀!什么缘故呢?斯陀含的意思是一往来,已证初果,要再一往天再上,再一来人间,断除欲界思惑。而事实上,他对于五欲六尘已不起贪爱了,应是体顺无为真如之理,在这无为真如之理上,那有往来之相呢?因为他已无往来之相,所以才叫做斯陀含。」「须菩提!你认为阿那含能有这样的心念吗?『我已得阿那含果?』」「不会的,佛陀!为什么?阿那含的意思是不来,二果斯陀含,断除了欲界思惑以后,就永久居住于色界的四禅天,享受天上的福乐,不再来人间,所以才名为不来。所以心中已没有来不来的分别。因此,才称为阿那含。若他尚有证果之念,便是著了不来之相,就不可以称为阿那含。」「须菩提!阿罗汉能起一种念头?『我已证得阿罗汉果!』」「不会的,佛陀!怎么说呢?因为实际上并没有什么法叫做阿罗汉。所谓的阿罗汉是彻悟我、法二空,不再随妄境动念,只是寂然如如,才为此立一假名。佛陀!如果阿罗汉起了我得阿罗汉的念头,那么,就是有了我、人、众生、寿者等法相对待分别,就不可以称为阿罗汉。「佛陀!您说我已证得无诤三昧,是人中第一,亦为罗汉中第一离欲的阿罗汉。但我并没有执著我是离欲罗汉的念头。佛陀!如果我有得阿罗汉道的念头,佛陀就不会称我为阿罗汉,那么,佛陀也不会赞歎我是欢喜修阿兰那行。因为须菩提并不存有修行的心相,妄念不生,所以才称为是欢喜修阿兰那行的。」}

\chapter{第十品\ 庄严净土分}
佛告须菩提:“于意云何,如来昔在然灯佛所,于法有所得不?”“不也,世尊,如来在然灯佛所,于法实无所得。”“须菩提,于意云何,菩萨庄严佛土不?”“不也,世尊。何以故?庄严佛土者,即非庄严,是名庄严。”“是故,须菩提,诸菩萨摩诃萨,应如是生清净心,不应住色生心,不应住声、香、味、触、法生心,应无所住,而生其心。须菩提,譬如有人,身如须弥山王,于意云何?是身为大不?” 须菩提言:“甚大,世尊。何以故?佛说非身,是名大身。”

\kai{佛陀再问须菩提:「你认为如何?佛陀以前在然灯佛时,有没有得到什么成佛的妙法?」「没有的,佛陀!因为诸法实相,本来清净具足,没有什么可说,也没有什么可得的成佛妙法。如果有所得的心,就无法和真如实相相契合。」佛陀颔首微笑,因为须菩提已领悟了真空无相法的真谛。于是,佛陀接著问道:「须菩提!你认为如何?菩萨有没有庄严佛土呢?」「没有的,佛陀!为什么呢?菩萨庄严佛土,只是权设方便,度化众生,若存有庄严清净佛土的心念,便是著相执法,就不是清净心。著相的庄严佛土,便落入世间的有漏福德,即非真正庄严佛土。庄严二字,只是为了度化众生,权立一个名相而已。」「所以,须菩提!诸位大菩萨都应该像这样生起清净心,不应该对眼识所见的种种色相生起迷恋、执著,也不应该执迷于声香味触法等尘境,应该心无所住,令清净自心显露。「须菩提!譬如有一个人,他的身体像须弥山王那样高大,你认为如何?他这个身体大不大?」须菩提回答道:「很大的,佛陀!为什么呢?佛陀所说的不是无相的法身,是指有形色、大小的色身,因此称这身体为大。如果以法身而言,是不可丈量,当然不是世间大小分别所能涵盖的。」}

\chapter{第十一品\ 无为福胜分}
“须菩提,如恒河中所有沙数,如是沙等恒河,于意云何?是诸恒河沙,宁为多不?”须菩提言:“甚多,世尊。但诸恒河,尚多无数,何况其沙。”“须菩提,我今实言告汝,若有善男子善女人,以七宝满尔所恒河沙数三千大千世界,以用布施,得福多不?” 须菩提言:“甚多,世尊。” 佛告须菩提:“若善男子善女人,于此经中,乃至受持四句偈等,为他人说,而此福德,胜前福德。”

\kai{「须菩提!像恒河中所有沙数,每一粒沙又成一恒河,这么多的恒河沙数,你认为算不算多呢?」须菩提回答:「太多了,佛陀!如果以一粒沙表示一个恒河,恒河尚且无法计数,何况是恒河里的沙数呢?」「须菩提!我现在实实在在的告诉你,如果有善男子、善女人,拿了七宝积满恒河沙数那样多的三千大千世界来布施,他们所获得的福德多不多呢?」须菩提回答:「非常多,佛陀!」佛陀进一步告诉须菩提:「如果有善男子、善女人,对这部《金刚经》能够信受奉持,甚至只是受持四句偈等,能够将经义向他人解说,使别人也对这部经生起无限信仰之心。那么,这个法施的福德胜过七宝布施的福德。
}
\chapter{第十二品\ 尊重正教分}
“复次,须菩提,随说是经,乃至四句偈等,当知此处,一切世间天、人、阿修罗,皆应供养,如佛塔庙。何况有人,尽能受持、读诵。须菩提,当知是人,成就最上第一希有之法。若是经典所在之处,即为有佛,若尊重弟子。”

\kai{其次,须菩提!不论什么人,什么处所,只要是解说这部《金刚经》,甚至只是经中的四句偈而已,这个说经的地方,一切世间,所有的天、人、阿修罗等,都应该前来护持、恭敬恭养,就如同供养佛的塔庙一样,更何况有人能尽他自己的所能,对这部经义信受奉行、读诵受持。须菩提!你们应当知道,这样的人已成就了最上第一希有的妙法。这部经典所在的地方,就是佛的住处,应当恭敬恭养。并且应尊重佛陀的一切弟子,因为有佛陀的地方,必定有圣贤弟子大众随侍左右。」}

\chapter{第十三品\ 如法受持分}
尔时须菩提白佛言:“世尊,当何名此经?我等云何奉持?” 佛告须菩提:“是经名为金刚般若波罗蜜,以是名字,汝当奉持。所以者何?须菩提,佛说般若波罗蜜,即非般若波罗蜜,是名般若波罗蜜。须菩提,于意云何,如来有所说法不?” 须菩提白佛言:“世尊,如来无所说。”“须菩提,于意云何?三千大千世界所有微尘,是为多不?” 须菩提言:“甚多,世尊。”“须菩提,诸微尘,如来说非微尘,是名微尘。如来说世界,非世界,是名世界。须菩提,于意云何,可以三十二相见如来不?”“不也,世尊,不可以三十二相得见如来。何以故?如来说三十二相,即是非相,是名三十二相。”“须菩提,若有善男子、善女人,以恒河沙等身命布施,若复有人,于此经中,乃至受持四句偈等,为他人说,其福甚多。”

\kai{这时候,须菩提请示佛陀说道:「佛陀!这部经应当如何称呼呢?我们应当如何信受奉持?」佛陀告诉须菩提:「这部经的名字就叫做《金刚般若波罗蜜》,真如法性如金刚之坚固猛利,不为物所摧毁,以此名称,你应当奉持。为什么呢?须菩提!佛陀所说的般若波罗蜜,为令众生迷途知返,离苦得乐,因此立此假名,随应众生机缘说法,其实并非有般若可以取著。只因为法本无说,心亦无名。」「须菩提!你认为如何?如来有所说法吗?」须菩提回答道:「佛陀!如来无所说法。」「须菩提!你以为三千大千世界的所有微尘,算不算多呢?」须菩提回答说:「非常多,佛陀!」「须菩提!这些微尘,毕竟也只是因缘聚合的假相,所以如来说这些微尘,不是具有真实体的微尘,只是假名叫做微尘而已。如来所说的三千大千世界也是缘成则聚,缘尽则灭,空无自性,不是真实不变的,只是假名为世界而已。」「须菩提!你认为如何?可不可以从三十二相上见到如来呢?」「不可以的,佛陀!不可以从三十二相上见如来的真实面目。为什么呢?如来所说的三十二相。应身为度化众生而出现的因缘假相。所以,不是如来真实的法身理体,只是假名为三十二相而已。」「须菩提!如果有善男子、善女人,用恒河沙数的身命来布施;又有人只从这部经典信受奉持,甚至只是经中的四句偈而已,并且为他人解说,使其明了自性,他所得的福德远胜过用身命布施的人。」}

\chapter{第十四品\ 离相寂灭分}
尔时须菩提闻说是经,深解义趣,涕泪悲泣,而白佛言:“希有,世尊。佛说如是甚深经典,我从昔来所得慧眼,未曾得闻如是之经。世尊,若复有人,得闻是经,信心清净,即生实相。当知是人,成就第一希有功德。世尊,是实相者,即是非相,是故如来说名实相。世尊,我今得闻如是经典,信解受持不足为难;若当来世后五百岁,其有众生,得闻是经,信解受持,是人即为第一希有。何以故?此人无我相、人相、众生相、寿者相。所以者何?我相即是非相,人相、众生相、寿者相,即是非相。何以故?离一切诸相,则名诸佛。” 佛告须菩提:“如是,如是。若复有人,得闻是经,不惊、不怖、不畏,当知是人,甚为希有。何以故?须菩提,如来说第一波罗蜜,即非第一波罗蜜,是名第一波罗蜜。须菩提,忍辱波罗蜜,如来说非忍辱波罗蜜。何以故?须菩提,如我昔为歌利王割截身体,我于尔时,无我相、无人相、无众生相,无寿者相。何以故?我于往昔节节支解时,若有我相、人相、众生相、寿者相,应生嗔恨。须菩提,又念过去于五百世,作忍辱仙人,于尔所世,无我相、无人相、无众生相、无寿者相。是故,须菩提,菩萨应离一切相,发阿耨多罗三藐三菩提心,不应住色生心,不应住声、香、味、触、法生心,应生无所住心。若心有住,即为非住。是故佛说菩萨心,不应住色布施。须菩提,菩萨为利益一切众生故,应如是布施。如来说一切诸相,即是非相,又说一切众生,即非众生。须菩提,如来是真语者、实语者、如语者、不诳语者、不异语者。须菩提,如来所得法,此法无实无虚。须菩提,若菩萨心住于法而行布施,如人入闇,即无所见;若菩萨心不住法而行布施,如人有目,日光明照,见种种色。须菩提,当来之世,若有善男子善女人,能于此经受持读诵,即为如来,以佛智慧,悉知是人,悉见是人,皆得成就无量无边功德。”

\kai{这时候,须菩提听闻了这部经的妙义,深深的了悟金刚经的义理旨趣,感激涕零地向佛陀顶礼赞歎,并请示佛陀说道:「世上希有的佛陀!佛陀所说的甚深微妙的经典,是我证得阿罗汉果,获得慧眼以来,还未曾听闻到的。佛陀!如果有人听闻了这经法,而能信心清净,那么,他便有了悟实相的智能,应当知道这人已经成就了第一希有的功德。佛陀!实相即是非一切相,所以如来说以非一切相之本相,不执求、不住著,即名为实相。「佛陀!我今日能够亲闻佛陀讲这部经典,能够信解受持,这并不是难事,若是到了末法时代,最后五百年,如果有众生,在那时听闻这微妙经义,而能够信心清净信受奉持,这个人便是世上第一希有的人。为什么呢?因为这人已顿悟真空之理,没有我、人、众生、寿者等四相的分别了。为什么呢?因为这四相本非真实,如果能离这些虚妄分别的幻相,那么,就没有我、人、众生、寿者等四相的执著了。为什么呢?远离一切虚妄之相,便与佛无异,而可以称之为佛了。」佛陀见须菩提已深解义趣,便为他印可道:「很好!很好!如果有人听闻这部经,而对于般若空理能够不惊疑、不恐怖、不生畏惧,应当知道,这人是非常甚为希有难得的。为什么呢?须菩提!因为他了悟了如来所说的第一波罗蜜,即不是第一波罗蜜,因六波罗蜜性皆平等,无高低次第,并没有所谓的第一波罗蜜。五波罗蜜,皆以般若为导,若无般若,就如人无眼,所以,第一波罗蜜只是方便的假名而已。「须菩提!忍辱波罗蜜,如来说非实有忍辱波罗蜜,因为般若本性,是寂然不动的,哪有忍辱不忍辱的分别?所以,忍辱波罗蜜也只是度化众生的假名而已。为什么呢?须菩提!我过去受歌利王节节支解身体,我当时,因得二空般若智,没有我法二执,所以,没有我、人、众生、寿者等四相的执著。为什么呢?当时我的身体被节节支解时,如果有我、人、众生、寿者等四相的执著,便会生起瞋恨心。「须菩提!我回想起我在修行忍辱波罗蜜的五百世中,在那时,内心也无我、人、众生、寿者四相等的执著,所以能慈悲忍辱,不生瞋恨。所以,须菩提!菩萨应该舍离一切妄相,发无上正等正觉的菩提心,不应该住于色尘上生心,也不应该住于声、香、味、触、法等诸尘上生心,应当无所执著而生清净心。如果心有所住,便会随境而迷,就无法无住而生其心了。所以佛陀说:菩萨不应该有任何事相上的执著,而行布施。「须菩提!菩萨发心为了利益一切众生,便应该如此不住相布施。如来说,一切相无非是邪计谬见、业果虚妄之假相,所以一切相即非真相,不过是因缘聚合的幻现而成,非有非空。又说,一切众生是地、水、火、风四大因缘聚合而成,生灭变化,不应著有,不应著空,应无所执著。所以一切众生即不是众生。「须菩提!如来所说的法是不妄的、不虚的、如所证而语的、不说欺诳的话。「须菩提!如来所证悟的法,既非实又非虚无。须菩提!如果菩萨心里执著有一个可布施的法而行布施,那就像一个人掉入黑暗中一样,一无所见。如果菩萨心能不住法而行布施,就像人有眼睛,在日光下洞见一切万物。「须菩提!未来之时,如果有善男子、善女人,能从这部经信受奉行、讽诵受持,即为如来以佛的智能,悉知悉闻悉见这人,成就无量无边无尽的功德。」}

\chapter{第十五品\ 持经功德分}
“须菩提,若有善男子善女人,初日分以恒河沙等身布施,中日分复以恒河沙等身布施,后日分亦以恒河沙等身布施,如是无量百千万亿劫,以身布施。若复有人,闻此经典,信心不逆,其福胜彼。何况书写、受持、读诵、为人解说。须菩提,以要言之,是经有不可思议,不可称量,无边功德,如来为发大乘者说,为发最上乘者说,若有人能受持读诵、广为人说,如来悉知是人、悉见是人,皆得成就不可量、不可称、无有边、不可思议功德,如是人等,即为荷担如来阿耨多罗三藐三菩提。何以故?须菩提,若乐小法者,著我见、人见、众生见、寿者见,则于此经,不能听受、读诵、为人解说。须菩提,在在处处,若有此经,一切世间天、人、阿修罗,所应供养,当知此处,则为是塔,皆应恭敬,作礼围绕,以诸华香而散其处。”

\kai{「须菩提!如果有善男子、善女人,在早晨时,以等于恒河沙等身布施;中午时,又以恒河沙等身布施;夜晚时,也以恒河沙等身命布施。如此一天三次布施,经过了百千万亿劫都没有间断过,这个人所得的福德,是难以计量。但是,如果一个人,他只是听闻此经之经义,诚信不疑,悟得般若真理,发心依教修持,那么他所得的福德,胜过以身命布施的人。又何况将此经书写、受持、读诵,为他人解说的人,他不但明了自己的本性,更使他人见性,所得福德,就更加不可胜数了!「须菩提!总而言之,这部经所具的功德之大,不是心所能思,口所能议,秤所能称,尺所能量的,它重过须弥,深逾沧海,不但功德大,而且义理深,是如来独为发大乘菩萨道心以及发最上佛乘的众生而说的!如果有人能受持读诵《金刚般若经》,并且广为他人说法,如来会完全知道此人,并眼见此人,皆能够成就不可称量、无有边际、不可思议的功德。唯有这等具备般若智能,而又能读诵解说经义的行者,才能承担如来『无上正等正觉』的家业。为什么呢?须菩提!一般乐于小法的二乘人,执著我见、人见、众生见、寿者见,对于此部大乘无相无住的妙义,是无法相信接受的,更不愿读诵,更不用说为他人解说了。「须菩提!般若智能在人人贵,在处处尊,所以不论何处,只要有这部经的地方,一切世间天、人、阿修罗等都应当恭敬供养。应当知道,此经所在之处,即是塔庙,一切众生都要恭敬地顶礼围绕,以芳香的花朵散其四周,虔诚地供养。}

\chapter{第十六品\ 能净业障分}
“复次须菩提,善男子善女人,受持、读诵此经,若为人轻贱,是人先世罪业,应堕恶道,以今世人轻贱故,先世罪业则为消灭,当得阿耨多罗三藐三菩提。须菩提,我念过去无量阿僧祇劫,于然灯佛前,得值八百四千万亿那由他诸佛,悉皆供养承事,无空过者。若复有人,于后末世,能受持读诵此经,所得功德,于我所供养诸佛功德,百分不及一,千万亿分,乃至算数譬喻所不能及。须菩提,若善男子善女人,于后末世,有受持读诵此经,所得功德,我若具说者,或有人闻,心则狂乱,狐疑不信。须菩提,当知是经义不可思议,果报亦不可思议。”

\kai{「再说,须菩提!如果有善男子、善女人一心修持读诵此经,若不得人天的恭敬,反而受人讥骂或是轻贱,那是因为此人先世所造的罪业很重,本应堕入三恶道中去受苦,但是他能在受人轻贱之中,依然不断地忍辱修持,了知由过去惑因而造下恶业,今信受此经,由于信心清净,便知惑业亦空,就可使宿业渐渐消灭,将来证得无上正等正觉。「须菩提!我回想起过去无数劫前,在然灯佛处,值遇八百四千万亿那由他诸佛,都一一亲承供养,一个也没有空过。假使有人,在末法之中,能诚心地受持读诵此经,所得的功德,和我所供养诸佛的功德相较,我是百分不及一,千万亿分不及一,甚至是算数、譬喻所无法相比的。「须菩提!若有善男子、善女人,于末法之中,受持读诵此经,所得的功德之多,我如果一一具实说出,或者有人听我说这些功德,其心会纷乱如狂,狐疑而不相信。须菩提!为什么有人听了会这样心智狂乱呢?那是因为这部经的义理甚深,不可思议,所以持受它所得的果报也就不可思议。」}


\chapter{第十七品\ 究竟无我分}
尔时须菩提白佛言:“世尊,善男子善女人,发阿耨多罗三藐三菩提心,云何应住?云何降伏其心?” 佛告须菩提:“善男子善女人,发阿耨多罗三藐三菩提心者,当生如是心,我应灭度一切众生;灭度一切众生已,而无有一众生实灭度者。何以故?须菩提,若菩萨有我相、人相、众生相、寿者相,则非菩萨。所以者何?须菩提,实无有法,发阿耨多罗三藐三菩提心者。须菩提,于意云何,如来于然灯佛所,有法得阿耨多罗三藐三菩提不?”“不也,世尊,如我解佛所说义,佛于然灯佛所,无有法得阿耨多罗三藐三菩提。” 佛言:“如是,如是,须菩提,实无有法,如来得阿耨多罗三藐三菩提。须菩提,若有法,如来得阿耨多罗三藐三菩提者,然灯佛即不与我授记:‘汝于来世当得作佛,号释迦牟尼。’以实无有法,得阿耨多罗三藐三菩提,是故然灯佛与我授记,作是言:‘汝于来世,当得作佛,号释迦牟尼。’何以故?如来者,即诸法如义。若有人言,如来得阿耨多罗三藐三菩提,须菩提,实无有法,佛得阿耨多罗三藐三菩提。须菩提,如来所得阿耨多罗三藐三菩提,于是中无实无虚。是故如来说一切法,皆是佛法。须菩提,所言一切法者,即非一切法,是故名一切法。须菩提,譬如人身长大。” 须菩提言:“世尊,如来说人身长大,即为非大身,是名大身。”“须菩提,菩萨亦如是。若作是言:‘我当灭度无量众生。’则不名菩萨。何以故?须菩提,实无有法,名为菩萨。是故佛说,一切法无我、无人、无众生、无寿者。须菩提,若菩萨作是言:‘我当庄严佛土。’是不名菩萨。何以故?如来说庄严佛土者,即非庄严,是名庄严。须菩提,若菩萨通达无我法者,如来说名真是菩萨。”

\kai{这时候,须菩提向佛陀请示道:「佛陀!善男子、善女人,已经发心求无上正等正觉,应该如何保持那颗菩提心?如何降伏那妄想动念的心?」佛陀了解须菩提再次启请的深意,微笑颔首之后,以无上慈和的声音说道:「善男子、善女人如果已经发心求无上正等正觉,应当如是发心:我应该发起无上清净心,使众生灭除一切烦恼,到达涅槃的境界,如此灭度一切众生,但不认为有一个众生是因我而灭度的。为什么呢?须菩提!如果菩萨有我相、人相、众生相、寿者相等分别,那么,他就不是菩萨。为什么呢?须菩提!实际上,并没有一种法名为发心求无上正等正觉的。「须菩提!你认为如何?当年佛陀在然灯佛那里,有没有得到一种法叫做无上正等正觉的?」须菩提回答道:「没有的,佛陀!依我听闻佛陀所讲的意义,我知道佛陀在然灯佛那里,只是了悟诸法空相,所以没有得到一种法叫做无上正等正觉的。」佛陀听完须菩提肯定的答复后,喜悦地说道:「很好!须菩提!实际上,我并没有得到一种法叫做无上正等正觉的。须菩提!如果我有得到一种法叫做无上正等正觉,然灯佛就不会为我授记说:『你在来世,一定作佛,名释迦牟尼。』正因为没有所谓的无上正等正觉之法可得,所以然灯佛才为我授记:你在来世,一定作佛,名叫释迦牟尼。「为什么呢?所谓如来,就是一切诸法体性空寂,绝对的平等,超越所有差别的执著。佛陀已证入此理,因此才名为如来。如果有人说,我得了『无上正等正觉』。须菩提!实际上并没有一种法,叫做佛得到无上正等正觉,只是为了令众生明了修行的趣向,方便设有无上正等正觉的假名。「须菩提!我所得无上正等正觉,是虚实不一,不能执为实有所得,也不能执为空无,因为一切诸法万象,无一不是从此空寂性体所显现的,所以,如来说一切诸法都是佛法。「须菩提!所说一切法,只是就随顺世谛事相而言,就空寂性体的立场,一切万事万物,都不是真实的,以此显发的事相,而立种种假名。「须菩提!譬如人身长大。」须菩提回答道:「佛陀!您说过:『这高大健壮的人身,毕竟是个无常虚假的形相,缘聚则成,缘尽则灭,所以不是大身,只是假名大身而已。』法身无相,又哪里有大小形状呢?」「须菩提!菩萨也应当明白这些道理,如果作是说:『我当灭度无量的众生。』他就不是菩萨。为什么呢?须菩提!实际上没有一个法名为菩萨,如果有当度众生的想法时,就有人我的妄执,能度所度的对待,所以我说一切诸法,都没有我、没有人、没有众生、没有寿者等四法的分别。「须菩提!如果菩萨作是说:『我当庄严佛土。』就不能名为菩萨,因为落入凡夫的我见法执。为什么呢?佛陀说的,庄严佛土,并不是有一真实的佛土可庄严,只是为了引度众生,修福积慧,涤除内心的情念妄执,而假名庄严佛土。「须菩提!如果菩萨通达无我的真理,那么,如来说他是真正的菩萨。}

\chapter{第十八品\ 一体同观分}
“须菩提,于意云何,如来有肉眼不?”“如是,世尊,如来有肉眼。”“须菩提,于意云何,如来有天眼不?”“如是,世尊,如来有天眼。”“须菩提,于意云何,如来有慧眼不?”“如是,世尊,如来有慧眼。”“须菩提,于意云何,如来有法眼不?”“如是,世尊,如来有法眼。”“须菩提,于意云何,如来有佛眼不?”“如是,世尊,如来有佛眼。”“须菩提,于意云何,如恒河中所有沙,佛说是沙不?”“如是,世尊,如来说是沙。”“须菩提,于意云何,如一恒河中所有沙,有如是沙等恒河,是诸恒河所有沙数,佛世界如是,宁为多不?”“甚多,世尊。” 佛告须菩提:“尔所国土中,所有众生,若干种心,如来悉知。何以故?如来说诸心,皆为非心,是名为心。所以者何?须菩提,过去心不可得,现在心不可得,未来心不可得。”

\kai{阐发了究竟无我的义理之后,才能见万法如一,见众生心如我心。「须菩提!你认为如来有肉眼吗?」须菩提答:「有的,佛陀,如来有肉眼。」佛陀又问:「如来有天眼吗?」「是的,佛陀!如来也有天眼。」「须菩提!如来有慧眼吗?」「是的,如来具有慧眼。」「如来有没有法眼?」「是的,如来具有法眼。」「须菩提!如来具有遍照一切十界的佛眼吗?」「是的,佛陀!如来有佛眼。」「须菩提!你认为,恒河中的所有沙粒,如来说是不是沙?」「是的,如来说是沙。」「须菩提!如果一沙一世界,那么像一条恒河沙那么多的恒河,这河中每一粒沙都代表一个佛世界的话,如此,佛世界算不算多?」「很多的,佛陀!」佛陀又问:「须菩提!如你刚才所说,佛眼可摄一切眼,一沙可摄一切沙,在诸佛世界中的一切众生,所有种种不同的心,佛也是完全知晓的。为什么呢?因为众生的心源与佛如一,譬如水流歧脉,源头是一,心性同源,众生心即是佛心,所以,如来能悉知众生心性。但是,众生往还六道,随业逐流,遗失了本心,反被六尘的妄想心所蒙蔽,生出种种虚妄心念,这种种心皆不是真实不变的心性,只是一时假名为心而已。这过去之心、现在之心、未来之心,无非皆由六尘缘影而生,念念相续,事过则灭,这种种无常虚妄之心,是不可得的。}

\chapter{第十九品\ 法界通化分}
“须菩提,于意云何,若有人满三千大千世界七宝,以用布施,是人以是因缘,得福多不?”“如是,世尊,此人以是因缘,得福甚多。”“须菩提,若福德有实,如来不说得福德多,以福德无故,如来说得福德多。”

\kai{须菩提!如果有人拿了满三千大千世界的七宝来布施的话,你想,这个人以是因缘,他得到的福报多不多呢?」「是的,佛陀!这个人以是因缘,得福很多。」「须菩提!如果福德有实在的体性,那么,我也就不会说得福德多了。正因为以不可得心为因,用七宝作缘,以如是因,如是缘,所以我才说得福德多。}

\chapter{第二十品\ 离色离相分}
“须菩提,于意云何,佛可以具足色身见不?”“不也,世尊,如来不应以具足色身见。何以故?如来说具足色身,即非具足色身,是名具足色身。”“须菩提,于意云何,如来可以具足诸相见不?”“不也,世尊,如来不应以具足诸相见。何以故?如来说诸相具足,即非诸相具足,是名诸相具足。”

\kai{「须菩提!你认为,佛可以从具足色身见到吗?」「不可以的,佛陀!不应该从圆满庄严的色身之处去见如来。为什么呢?因为如来说过,圆满报身,只是因缘假合的幻相,缘尽则灭,不是真实不变的实体,只是假名为色身而已。」「须菩提!可以从具足诸相中见到如来吗?」「不可以的,佛陀!不应从三十二相、八十种好之处去见如来。为什么呢?因为如来所说的诸相具足,是性德圆满而示现的幻象,是为了度化众生才显现的,并非真实的相貌,不过是一时的假名罢了。」}

\chapter{第二十一品\ 非说所说分}
“须菩提,汝勿谓如来作是念:我当有所说法。莫作是念,何以故?若人言如来有所说法,即为谤佛,不能解我所说故。须菩提,说法者,无法可说,是名说法。” 尔时慧命须菩提白佛言:“世尊,颇有众生,于未来世,闻说是法,生信心不?” 佛言:“须菩提,彼非众生,非不众生。何以故?须菩提,众生,众生者,如来说非众生,是名众生。”

\kai{「须菩提!你不要认为我有这样的意念:『我当有所说法。』你不可有如此生心动念。为什么呢?如果有人说如来『有所说法』的念头,那是毁谤佛陀,因为他不能了解我所说之故。「须菩提!一切言说是开启众生本具的真如自性,为了袪除众生妄念,随机化度,随缘而说,何来有法?这种种言声的说法,也只是一时的方便言语,暂且给它一个『说法』的假名。」这时候,深具智能的须菩提了解佛陀的深意,但又怕末世众生听闻无法可说,无说法者,这番言语,狐疑不信,于是,便请问佛陀道:「佛陀!将来的众生听了您今日『无说而说』的妙义之后,能生起信心吗?」佛陀当下便斧底抽薪,破除弟子们对佛与众生们的分别见,说:「须菩提!他们既不是众生,也不能说不是众生。为什么呢?就法性空寂而言,他们也是佛,是尚未了悟真理的佛。佛也是众生,是已悟道的众生。但是,又不能不称之为众生,因为他们虽已经听闻佛法,生起信心,但还未能悟道,所以于事相上说,称他们为众生。须菩提!从真如本性上来说,众生即佛,原来没有什么众生不众生的,『众生』也只是一时的假名而已。」}

\chapter{第二十二品\ 无法可得分}
须菩提白佛言:“世尊,佛得阿耨多罗三藐三菩提,为无所得耶?” 佛言:“如是,如是,须菩提,我于阿耨多罗三藐三菩提,乃至无有少法可得,是名阿耨多罗三藐三菩提。”

\kai{须菩提心有所悟,向佛陀说:「佛陀!您得无上正等正觉,是真无所得!」佛陀印可说:「是的,须菩提!不仅是无上正等正觉,乃至纤毫之法,我都无所得。得者,因为有失也,我本无所失,何来有得?无上正等正觉之名,指的是觉悟,自性,而非有所得。}

\chapter{第二十三品\ 净心行善分}
“复次,须菩提,是法平等,无有高下,是名阿耨多罗三藐三菩提。以无我、无人、无众生、无寿者,修一切善法,即得阿耨多罗三藐三菩提。须菩提,所言善法者,如来说即非善法,是名善法。”

\kai{其次,须菩提!人不分贤愚圣凡,其真如菩提绝对平等的,没有高下的分别,所以才名为无上正等正觉。只要众生不执著于我相、人相、众生相、寿者相的妄想分别去修持一切善法,那么即可悟得无上正等正觉。须菩提!所谓的善法,也不过是因缘和合的假象,怎能执为实有?修一切善法,不可著相,善法之名,不过是随顺世俗事相而言。}

\chapter{第二十四品\ 福智无比分}
“须菩提,若三千大千世界中,所有诸须弥山王,如是等七宝聚,有人持用布施。若人以此般若波罗蜜经,乃至四句偈等,受持读诵,为他人说,于前福德,百分不及一,百千万亿分,乃至算数譬喻所不能及。”

\kai{须菩提!如果以三千大千世界中,所有须弥山王作比较,有人用七宝,集满所有的须弥山王,用来布施,这个人所得的福德,当然是很多的。但是如果有人只是受持读诵这部《金刚般若波罗蜜经》,并且又能为他人解说,哪怕只有四句偈,他所得的福德,用七宝布施的福德校量,前者的布施福德,是百不及一,百千万亿分不及一,甚至是算数譬喻所不能相比的。}

\chapter{第二十五品\ 化无所化分}
“须菩提,于意云何,汝等勿谓如来作是念:‘我当度众生。’须菩提,莫作是念。何以故?实无有众生如来度者;若有众生如来度者,如来即有我、人、众生、寿者。须菩提,如来说有我者,即非有我,而凡夫之人,以为有我。须菩提,凡夫者,如来说即非凡夫,是名凡夫。”

\kai{佛陀恐怕还有众生以为他有众生可度,所以特地再一次提出询问:「须菩提!你不要说,我还有『众生可度』的念头,你不要有这样的想法。为什么呢?因为众生当体即空,并无实在之相,如果我还生心动念,有众生可度,那么连我自己也落入我、人、众生、寿者四相的执著之中。「须菩提!如来所说的『我』,事实上是假相的我,是为了度化众生,权巧方便设立的,但是凡夫却以为有个真实的我,这都是凡夫执相成迷。「须菩提!其实以心、佛、众生三无差别,一切凡夫都具有如来智能,凡夫与佛,本来平等的,所以凡夫并非凡夫,只是因为他一时沈沦不觉,随逐妄缘,未能了悟生死,暂时假名为凡夫。}

\chapter{第二十六品\ 法身非相分}
“须菩提,于意云何,可以三十二相观如来不?” 须菩提言:“如是如是,以三十二相观如来。” 佛言:“须菩提,若以三十二相观如来者,转轮圣王即是如来。” 须菩提白佛言:“世尊,如我解佛所说义,不应以三十二相观如来。” 尔时世尊而说偈言:若以色见我,以音声求我,是人行邪道,不能见如来。

\kai{「须菩提!你认为如何?可以从三十二相观如来吗?」须菩提自然知道佛陀这一问的深意,便从众生立场所见作答:「是的,佛陀!可以从三十二相观如来。」佛陀便接著须菩提的回答,一语道出「法身非相」的真理说:「须菩提!若能以三十二相观如来,那么转轮圣王也具足三十二相,他也是如来了。」须菩提心有领悟,立即回答:「佛陀!如我解悟佛陀所说之义,是不可以从三十二相观如来的。」这时候佛陀以偈说道:若有人想以色见我,以声音求我;此人心有住相,就是行邪道。}

\chapter{第二十七品\ 无断无灭分}
“须菩提,汝若作是念:‘如来不以具足相故,得阿耨多罗三藐三菩提。’ 须菩提,莫作是念:‘如来不以具足相故,得阿耨多罗三藐三菩提。’须菩提,汝若作是念,发阿耨多罗三藐三菩提心者,说诸法断灭,莫作是念。何以故?发阿耨多罗三藐三菩提心者,于法不说断灭相。”

\kai{佛陀一路破执至此,又怕众生落入断灭空见的陷阱之中,所以抽丝剥茧,好比一手推著,一手挡著,无非要众生当下自悟。须菩提!你不要有这样的念头,如来不以具足相的缘故,才得到无上正等正觉的。你决不可以认为,如来因不以具足相而得到无上正等正觉。须菩提!你如果生起这样的想法,发无上正等正觉菩提心,就会说诸法断灭,认为不须要有什么善法的修行。为什么呢?因为发无上正等正觉心的人,于法不说断灭相,不著法相,也不著断灭相。}

\chapter{第二十八品\ 不受不贪分}
“须菩提,若菩萨以满恒河沙等世界七宝,持用布施。若复有人,知一切法无我,得成于忍。此菩萨,胜前菩萨所得功德。何以故?须菩提,以诸菩萨不受福德故。” 须菩提白佛言:“世尊,云何菩萨,不受福德?”“须菩提,菩萨所作福德,不应贪著,是故说不受福德。”

\kai{须菩提!菩萨若用满恒河沙等世界的七宝来布施,所得功德,当然无法计量。如果明白一切法无我,皆由因缘所生,无有真实永恒的体性,由此了知无生无灭,不为外境所动,即与空性相应。内无贪念,外无所得,亲证无生法忍,那么,这位菩萨所得的功德要比七宝布施的菩萨更多的。「为什么呢?须菩提!因为诸菩萨是不受福德相的限制。」须菩提不解的问道:「什么是诸菩萨不受福德的限制呢?」「须菩提!菩萨所作福德,不应贪求生起执著。因为菩萨行利益众生事,是发菩提心,而不是贪求福德,是利他而非利己。菩萨修一切善法,行六度万行,不著相布施,心中并没有计较福德的妄念,所以才说菩萨不受福德相的限制。}


\chapter{第二十九品\ 威仪寂净分}
“须菩提,若有人言:‘如来若来、若去,若坐、若卧。’是人不解我所说义。何以故?如来者,无所从来,亦无所去,故名如来。”

\kai{须菩提!如果有人说,如来也是有来、去、坐、卧等相,这个人就是不了解我所说如来的深意了。为什么呢?所谓如来者,实在是无所来处,也无所去处,所以才称为如来。因为如来就是法身,法身无形无相,遍满虚空,无所不在,寂然不动,哪里还有来去之名呢?众生所见的语默动静之相,不过是如来的应化之身,应化身为随众生之机缘感应有隐有现,但是法身则恒常寂静,从未有来、去、坐、卧的相状。}

\chapter{第三十品\ 一合理相分}
“须菩提,若善男子善女人,以三千大千世界碎为微尘。于意云何,是微尘众,宁为多不?”须菩提言:“甚多,世尊。何以故?若是微尘众实有者,佛即不说是微尘众。所以者何?佛说微尘众,即非微尘众,是名微尘众。世尊,如来所说三千大千世界,即非世界,是名世界。何以故?若世界实有者,即是一合相。如来说一合相,即非一合相,是名一合相。”“须菩提,一合相者,则是不可说,但凡夫之人,贪著其事。”

\kai{须菩提!如果有善男子、善女人,把三千大千世界都碎成微尘,你认为这些微尘多不多呢?」「太多了,佛陀!为什么呢?如果这些微尘众,是实有恒常的体性,佛陀就不会说它多了。佛陀所说的微尘众,实是缘生的假相,并没有恒常不变的自性,只是一个假名而已。「佛陀!如来说过,三千大千世界并非即是真实恒常的世界,也仅是一个假名而已。为什么呢?如果世界是实有的,那就是一合相。如来说的一合相,也非实有,缘生则聚,分合离散,仍然不是实有不变的一合相,也只是缘散即无,一个假名罢了。」「须菩提!所谓一合相,没有定相可言,本是个众缘和合而有的,非空非有,如何可以言说?但是凡夫之人执著取相,贪恋执著有个真实的一合相。}

\chapter{第三十一品\ 知见不生分}
“须菩提,若人言:‘佛说我见、人见、众生见、寿者见。’须菩提,于意云何,是人解我所说义不?”“不也,世尊,是人不解如来所说义。何以故?世尊说我见、人见、众生见、寿者见,即非我见、人见、众生见、寿者见,是名我见、人见、众生见、寿者见。”“须菩提,发阿耨多罗三藐三菩提心者,于一切法,应如是知、如是见、如是信解,不生法相。须菩提,所言法相者,如来说即非法相,是名法相。”

\kai{「须菩提!如果有人说,佛陀宣说的我见、人见、众生见、寿者见,是真实的。须菩提!你认为这个人了解我所说的深意吗?」「佛陀!这个人不曾了解您所说的深意。为什么呢?佛陀说我见、人见、众生见、寿者见,都是虚妄不实的,只是随缘而设立的假名。众生迷于事相为有,若能悟知体性空寂则无,不可于此四见,妄执实有。」「须菩提!发无上正等正觉之心的人,对于一切世间法、出世间法,都应该如实去知,如实去见,如实去信解,心中不生一切法相,而妄起执著。「须菩提!你应当知道,所谓的法相,并非有真实不变的法相,只是缘起的幻相,佛陀暂时应机说法的假名而已。}

\chapter{第三十二品\ 应化非真分}
“须菩提,若有人以满无量阿僧祇世界七宝,持用布施。若有善男子善女人,发菩提心者,持于此经,乃至四句偈等,受持读诵,为人演说,其福胜彼。云何为人演说?不取于相,如如不动。何以故?{\color{red}一切有为法,如梦幻泡影,如露亦如电,应作如是观。}”佛说是经已,长老须菩提,及诸比丘、比丘尼、优婆塞、优婆夷,一切世间天、人、阿修罗,闻佛所说,皆大欢喜,信受奉行。

\kai{须菩提!如果有人,以充满无量阿僧只世界的七宝,以此为布施。如果有善男子、善女人发无上菩提心,受持这部《金刚经》,哪怕只有四句偈而已,他能信受读诵,且为他人解说,那么,他的福德自然要过胜过行七宝布施的人。要如何为他人演说呢?当不执著于一切相,随缘说法而如如不动。为什么呢?因为一切世间的有为诸法,就像梦境的非真,幻化的无实,水泡的易灭,影子的难存,又如早晨遇日而失的露珠,天空将雨时的闪电,瞬间即灭。应作如是的观照啊!」此时,佛陀说《金刚经》已经圆满了,长老须菩提,及同时在法会听经的比丘、比丘尼,优婆塞、优婆夷,一切世间的天、人、阿修罗等,听闻了佛陀说法之后,深深的了悟,无不法喜充满,一心信受奉行。}



\end{document}